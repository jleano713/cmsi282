\documentclass{article}

\usepackage{amsmath}
\usepackage{mathtools}
\usepackage{listings}
\usepackage[margin=1.0in]{geometry}

\title{CMSI 282 - Homework 2}
\author{Janine Leano}
\date{March 1st 2015}

\begin{document}
\maketitle

\begin{enumerate}
	\item 
		(a.) f = $\Theta$(g) \\
		(b.) f = O(g) \\
		(c.) f = $\Theta$(g) \\
		(d.) f = $\Theta$(g) \\
		(e.) f = $\Theta$(g) \\
		(f.) f = $\Theta$(g) \\
		(g.) f = $\Omega$(g) \\
		(h.) f = $\Omega$(g) \\
		(i.) f = $\Omega$(g) \\
		(j.) f = $\Omega$(g) \\
		(k.) f = $\Omega$(g) \\
		(l.) f = O(g) \\
		(m.) f = O(g) \\
		(n.) f = $\Theta$(g) \\
		(o.) f = $\Omega$(g) \\
		(p.) f = O(g) \\
		(q.) f = $\Theta$(g)
	\item
		(a.) \[
		\begin{bmatrix}
    			x_{11}       & x_{12} \\
    			x_{21}       & x_{22}  
		\end{bmatrix}
		\begin{bmatrix}
    			x_{11}       & x_{12} \\
    			x_{21}       & x_{22}  
		\end{bmatrix}
		=
		\begin{bmatrix}
    		x_{11} \times x_{11} + x_{12} \times x_{21} & x_{11} \times x_{12} + x_{12} \times x_{22}  \\
    		x_{21} \times x_{11} + x_{22} \times x_{21} & x_{21} \times x_{12} + x_{22} \times x_{22}
		\end{bmatrix}
		\] \\
		(b.) In order to get $X^{8}$, \\
		Start with $X^{2} = X^{2} \times X^{2} \\
		 X^{4} =  X^{2} \times  X^{2}  \\
		 X^{8} = X^{4} \times X^{4}$ \\
		 n is an exponential of 2 \\
		 In the general case $X^{n}$ where $n = 2^{k}$, at every iteration, n is doubled from before. The running time is O($\log(n)$) since it takes $k = \log_2(n)$ matrix multiplications in order to compute $X^{n}$.
		 
	\item
		N = number given \\
		d = number of digits in N \\
		In decimal: $10^{d - 1} = N \\
		d1 = \log_{10}(N) + 1$\\
		In binary: $d2 = \log_{2}(N) + 1\\
		\log_{2}(N) = \frac{\log_{10}(N)}{\log_{10}(2)} \leq 4$
	
	\item
		Upper bound: \\
		$n! = 1 \times 2 \times 3 \times \dots n \\
		n^{n} = n \times n \times n \times \dots n $\\
		with both of equal n lengths. $n! < n^{n}$ \\
		So $n! = O(n^{n})$ \\
		Lower bound: \\
		$n! = 1 \times 2 \times 3 \times \dots n \\
		(\frac{n}{2})^{\frac{n}{2}} = $ \\
		with both of equal n lengths. $n! > (\frac{n}{2})^{\frac{n}{2}} \\
		\log(n!) > \log(\frac{n}{2})^{\frac{n}{2}} \\
		\log(n!) > \frac{1}{2} \times n \times \log(\frac{1}{2} \times n) \\
		\frac{1}{2}s$ are constants so \\
		$\log(n!) > n\log(n)$ \\ \\
		Therefore, $\log(n!) = \Theta(n\log(n))$
		
	\item
		$4^{1536} \equiv 9^{4824}\mod(35)$, so yes
		
	\item
		According to Fermat's Theorem: \\
		$5^{30000}\equiv 1 \mod(31) \equiv 125 \mod(31) \equiv 6^{123456}$, so yes
		
	\item
		Given $b = 15$. Repeated squaring yields $2^{15} = 2^{8} \times 2^{4} \times 2^{2} \times 2^{1}$ (4 multiplications)\\
		Repeated $x^{5}$-ing yields $2^{15} = 2^{5} \times 2^{5} \times 2^{5}$ (3 multiplications)
		
	\item
		Using modular exponentiation, $2^{125}\mod(127) = 64$
		
	\item
		\lstinputlisting[language=Python]{cmsi282_homework2_question9.py}
		Running time is polynomial.
		
	\item
		\ We can't immediately base a primality test using Wilson's theorem because it is not efficient. It is harder to evaluate the larger the input. (source: Wikipedia article on Wilson's theorem)
		
	\item
		\lstinputlisting[language=Python]{cmsi282_homework2_question11.py}
	

\end{enumerate}

\end{document}